\documentclass{resume}

\newcommand{\en}[1]{#1}
\newcommand{\zh}[1]{}

\zh{\usepackage{xeCJK}}
\zh{\setCJKmainfont{Noto Sans CJK SC}}
\zh{\setCJKsansfont{Noto Sans CJK SC}}
\zh{\setCJKmonofont{Noto Sans CJK SC}}

\hypersetup{hidelinks}

\begin{document}
\linespread{1}

\name{\en{Songjian Li}\zh{李松健}}
\basicInfo{
      \email{career@songjian.li} \textperiodcentered\
      \phone{+86-17356375549} \textperiodcentered\
      \homepage[Blog]{https://abcdlsj.github.io/} \textperiodcentered\
      \github[GitHub]{https://github.com/abcdlsj/}
}

\section{\faGraduationCap\ \en{Education}\zh{教育经历}}
\datedsubsection{\textbf{\en{HeFei University of Technology}\zh{合肥工业大学}}}{09/2017 -- 07/2021}
\begin{itemize}
      \item \en{Bachelor's Degree in Electronic Information Science and Technology, \textit{School of Computer Science}}
            \zh{本科,电子信息科学与技术专业,\textit{计算机科学学院}}
\end{itemize}

\section{\faBriefcase\ \en{Work Experience}\zh{工作经历}}
\en{\datedsubsection{\textbf{\href{https://www.sea.com/products/shopee/}{Shopee Inc.}},ShenZheng,China}{07/2021 -- present}}
\zh{\datedsubsection{\textbf{\href{https://www.sea.com/products/shopee/}{深圳虾皮信息科技发展有限公司}}}{2021/07 -- 至今}}
\en{\role{Seller Platform \& Open Platform}{Backend Software Engineer}}
\zh{\role{卖家平台 \& Shopee 开放平台}{后端开发工程师}}
\begin{itemize}[parsep=0.5ex]
      \item \en{In the seller platform is responsible for the development and maintenance of the core modules of the main site (home page, pop-up windows, sidebars, feature-toggle), to improve the seller experience}
            \zh{在卖家平台负责开发和维护主站核心模块(首页、弹窗、侧边栏、功能开关),提升卖家及运营使用体验}
      \item \en{Mainly responsible for the development of developer-related functions in the open platform (log, analyze, search), to improve the experience of developers and operations}
            \zh{在开放平台主要负责开发者相关功能的开发(日志、分析、搜索),提升开发者及运营体验}
      \item \en{Participate in a variety of projects, including project refactoring, gateway plug-in development, middleware integration, standalone projects, common library maintenance, and seller account system construction}
            \zh{参与多元化项目,包括项目重构、网关开发、中间件集成、独立项目、Common库维护及卖家账号系统构建等}
      \item \en{Mainly use the Go language, interface protocols using HTTP/RPC, proficient application of common middleware (MySQL, Redis, Kafka, ElasticSearch, ClickHouse, etc)}
            \zh{主要使用 Go 语言,接口协议采用 HTTP/RPC,熟练应用常用中间件(MySQL,Redis,Kafka,ElasticSearch,ClickHouse,etc)}
      \item \en{Ability to design and implement solutions independently, good at cross-departmental coordination and communication, effectively manage project schedule and ensure on-time delivery}
            \zh{具有独立设计和实施解决方案的能力,善于跨部门协作与沟通,有效管理项目进度并确保按时交付}
\end{itemize}

\section{\faCloud\ \en{Projects}\zh{工作项目}}

\datedsubsection{\textbf{\en{Open Service}\zh{Open Service}}}{\url{https://open.shopee.com/}}
\begin{itemize}[parsep=0.5ex]
      \item \en{Developed with Gin/RPC, added key modules such as invocation dashboard, log search, document search, business analysis dashboard, vulnerability dashboard, etc}
            \zh{添加了调用看板、日志搜索、商业分析看板和漏洞看板等关键模块,增强了平台的功能性,提升了用户体验和平台可观测性}
      \item \en{Implemented OpenResty gateway plug-in for OpenAPI call log reporting (field desensitization, log structure normalization), data transfer through Kafka, write to HBase/ClickHouse/ElasticSearch after processing , to provide support for subsequent features}
            \zh{实现了 OpenResty 网关插件,用于开发者调用日志上报(字段脱敏,日志结构规范化),为其它功能提供支持}
\end{itemize}

\datedsubsection{\textbf{\en{Seller Centre}\zh{Seller Centre}}}{\url{https://seller.shopee.co.id/}}
\begin{itemize}[parsep=0.5ex]
      \item \en{Contains the main home page content, pop-up windows and sidebars and other modules, widely used asynchronous loading, caching and message queues and other technologies, significantly improving the site's responsiveness and seller experience}
            \zh{包含主站首页内容、弹窗和侧边栏等模块,广泛采用了异步加载、缓存和消息队列等技术,显著提升了网站的响应速度和卖家使用体验}
      \item \en{Feature toggle are used to flexibly control the display and limitation of functions, including multi-data source support, version rollback and multi-level query, etc., which facilitates efficient management of operations}
            \zh{功能开关用于灵活控制功能的显示和限制,包括多数据源支持、版本回滚和多级查询等功能,便于运营高效管理}
\end{itemize}

\datedsubsection{\textbf{\en{Partner Voucher}\zh{Partner Voucher}}}{\url{https://partnervoucher.shopee.co.id/}}
\begin{itemize}[parsep=0.5ex]
      \item \en{Developed by Gin, includes modules such as Order,User,Balance,Voucher management and so on}
            \zh{Partner 充值券码管理服务,包含 Order、User、Balance、Voucher 等模块,并提供相应的 Admin 功能}
      \item \en{Optimized the Partner recharge and redemption process to improve Ops efficiency. Independently completed the back-end design and development and coordinate with multiple teams, responsible for promoting the development part of the project and managing the progress.}
            \zh{提供在线充值兑换流程,提高了运营和用户效率。独立完成后端设计与开发并与多个团队联调,负责推进项目开发部分并管理进度}
\end{itemize}

\section{\faGithubAlt\ \en{Personal Projects}\zh{个人项目}}
\datedsubsection{\textbf{\en{Gnar}\zh{Gnar}}}{\url{https://github.com/abcdlsj/gnar}}
\en{A tool similar to frp that supports subdomain forwarding}
\zh{内网端口转发工具,类似于 Frp/Ngork}
\begin{itemize}[parsep=0.5ex]
      \item \en{Forwarding protocol support TCP/UDP, Crontol connection using TCP, using Yamux support multiplexing}
            \zh{转发协议支持 TCP/UDP,Crontol 连接采用 TCP,使用 Yamux 支持多路复用}
\end{itemize}

\datedsubsection{\textbf{\en{cLSM}\zh{cLSM}}}{\url{https://github.com/abcdlsj/clsm}}
\en{An implementation of LSM-Tree data structure based on SkipList}
\zh{基于 SkipList 的 LSM-Tree 数据结构实现}
\begin{itemize}[parsep=0.5ex]
      \item \en{Memory level uses SkipList data structure, disk level uses ordered key-value pairs}
            \zh{内存层使用 SkipList 数据结构,磁盘层使用有序键值对}
\end{itemize}

\section{\faCogs\ \en{Skills}\zh{技能}}
\begin{itemize}[parsep=0.5ex]
      \item \en{\textbf{Programming Languages}:Familiar with Go foundation,C++,Python,Lua,familiar with common data structures and algorithms}
            \zh{\textbf{编程语言}:熟悉 Go,了解 C++、Python、Lua,熟悉常用数据结构与算法}
      \item \en{\textbf{System \& Network}:Processes,memory,Linux,multiplexing,containerization techniques,common network protocols,etc}
            \zh{\textbf{系统 \& 网络}:进程、内存、Linux、多路复用、容器化技术、常见网络协议等}
      \item \en{\textbf{Milleware Framework}:Knowledge of commonly used middleware (MySQL/Redis/Kafka/ElasticSearch/ClickHouse), ability to choose reasonable middleware according to business needs and implementation}
            \zh{\textbf{中间件}:了解常用中间件及组件(MySQL/Redis/Kafka/ElasticSearch/ClickHouse),能根据业务需求选择合理的中间件并接入}
\end{itemize}

\section{\faInfo\ \en{Miscellaneous}\zh{杂项}}
\begin{itemize}[parsep=0.5ex]
      \item \en{Like to implement Small Side Project,includes Go Git,Tfidf wiki search engine,BitTorrent downloader,IM bot,Static-Blog generator,DNS server,line counter(Tokei/Scc),Proxy,Serverless Framework,Pastebin}
            \zh{喜欢实现 Small Side Project,包括 Go Git、Tfidf wiki 搜索引擎、BitTorrent downloader、IM bot、Static-Blog generator、line counter(Tokei/Scc)、Proxy、Serverless Framework、Pastebin 等等}
      \item \en{I love to explore, share and record.}
            \zh{喜欢折腾和探索,热爱分享和记录}
\end{itemize}

\end{document}