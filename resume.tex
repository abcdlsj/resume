\documentclass{resume}

\newcommand{\en}[1]{#1}
\newcommand{\zh}[1]{}

\zh{\usepackage{xeCJK}}
\zh{\setCJKmainfont{Noto Sans CJK SC}}
\zh{\setCJKsansfont{Noto Sans CJK SC}}
\zh{\setCJKmonofont{Noto Sans CJK SC}}

\hypersetup{hidelinks}

\begin{document}
\linespread{1}

\name{\en{Songjian Li}\zh{李松健}}
\basicInfo{
      \email{career@songjian.li} \textperiodcentered\
      \phone{+86-13225639902} \textperiodcentered\
      \homepage[Blog]{https://abcdlsj.github.io/} \textperiodcentered\
      \github[GitHub]{https://github.com/abcdlsj/}
}

\section{\faGraduationCap\ \en{Education}\zh{教育经历}}
\datedsubsection{\textbf{\en{HeFei University of Technology}\zh{合肥工业大学}}}{09/2017 -- 07/2021}
\begin{itemize}
      \item \en{Bachelor's Degree in Electronic Information Science and Technology, \textit{School of Computer Science}}
            \zh{本科,电子信息科学与技术专业,\textit{计算机科学学院}}
\end{itemize}

\section{\faBriefcase\ \en{Work Experience}\zh{工作经历}}
\en{\datedsubsection{\textbf{\href{https://www.sea.com/products/shopee/}{Shopee Inc.}},ShenZheng,China}{07/2021 -- present}}
\zh{\datedsubsection{\textbf{\href{https://www.sea.com/products/shopee/}{深圳虾皮信息科技发展有限公司}}}{2021/07 -- 至今}}
\en{\role{Seller Platform \& Open Platform}{Backend Software Engineer}}
\zh{\role{卖家平台 \& Shopee 开放平台}{后端开发工程师}}
\begin{itemize}[parsep=0.5ex]
      \item \en{As the Owner,Develop and maintain seller main site Home-Page(Content/Popups/Sidebars) and seller Feature-Toggle project}
            \zh{作为 Owner 开发和维护卖家主站首页(内容/弹窗/侧边栏)和卖家功能开关项目}
      \item \en{Participate in the development of Shopee open platform system, mainly responsible for the development of user experience related functions (logging, analyzing)}
            \zh{参与 Shopee 开放平台系统开发,主要负责提升用户体验相关功能(日志、分析、搜索)的开发}
      \item \en{Participate in project re-engineering, development team requirements(including OAuth,S3/DB migration,Gateway plug-in,Internal middleware,Common library,etc)}
            \zh{参与项目重构,开发团队其它需求,包括但不限于网关插件开发、OAuth 集成、S3/DB 数据迁移、内部框架对接、Common 库维护等}
      \item \en{Mainly use Go language, interface protocol using HTTP/RPC, skilled application of commonly used middleware (MySQL, Redis, Kafka, ElasticSearch, ClickHouse, etc), able to independently according to the needs of the design, select the appropriate solution and implementation}
            \zh{主要使用 Go 语言,接口协议采用 HTTP/RPC,熟练应用常用中间件(MySQL,Redis,Kafka,ElasticSearch,ClickHouse,etc),能够独立根据需求设计、选择合适的解决方案并实现}
\end{itemize}

\section{\faCloud\ \en{Projects}\zh{工作项目}}

\datedsubsection{\textbf{\en{Open Service}\zh{Open Service}}}{\url{https://open.shopee.com/}}
\begin{itemize}[parsep=0.5ex]
      \item \en{Add OpenResty gateway plug-in for OpenAPI call log reporting (field desensitization, structure normalization), data transfer through Kafka, consumption processing and write to HBase/ClickHouse/ElasticSearch, to provide support for subsequent functions}
            \zh{添加 OpenResty 网关插件,用于 OpenAPI 调用日志上报(字段脱敏、结构规范化),数据通过 Kafka 传输,消费处理后写入到 HBase/ClickHouse/ElasticSearch 中,为后续功能提供支持}
      \item \en{Developed with Gin/RPC, added key modules such as invocation dashboard, log search, document search, business analysis dashboard, vulnerability dashboard, etc}
            \zh{添加了调用看板、日志搜索、文档搜索、商业分析看板、漏洞看板等关键模块}
      \item \en{Improves the user experience and observability of the platform, and provides new requirement data support for the PM team}
            \zh{提升了用户体验和平台可观测性,为 PM 团队提供新需求数据支持}
\end{itemize}

\datedsubsection{\textbf{\en{Partner Voucher}\zh{Partner Voucher}}}{\url{https://partnervoucher.shopee.co.id/}}
\begin{itemize}[parsep=0.5ex]
      \item \en{Developed by Gin, includes modules such as Order,User,Balance,Voucher management and so on}
            \zh{Partner 充值券码管理服务,包含 Order、User、Balance、Voucher 等模块,并提供相应的 Admin 功能}
      \item \en{Optimized the Partner recharge and redemption process to improve Ops efficiency. Independently completed the back-end design and development and coordinate with multiple teams, responsible for promoting the development part of the project and managing the progress.}
            \zh{优化了 Partner 充值兑换流程,提高了 Ops 效率。独立完成后端设计与开发并与多个团队联调,负责推进项目开发部分并管理进度}
\end{itemize}

\datedsubsection{\textbf{\en{Seller Centre}\zh{Seller Centre}}}{\url{https://seller.shopee.co.id/}}
\begin{itemize}[parsep=0.5ex]
      \item \en{Developed modules for Home-Page,Pop-ups,Sidebars,Error-code,Tags,Feature-Toggle and Admin modules}
            \zh{负责主站首页内容、弹窗、侧边栏、错误码、标签、功能开关(Feature Toggle)等模块,以及对应的 admin 端功能}
      \item \en{Feature-Toggle control for seller,includes multi-data Source,Version,Rollback,multi-level Query and so on}
            \zh{功能开关用于卖家的功能 显示/限制,包括多数据源支持、版本、回滚、多级查询等功能}
      \item \en{The project is rich in content, the design function is clearly delineated, the business expansion is strong, and asynchronous loading, caching, messaging and other ways to improve the user experience are widely used.}
            \zh{项目内容丰富,设计功能划分清晰,业务拓展性强,广泛使用异步加载、缓存、消息等方式提高用户体验}
\end{itemize}

\section{\faGithubAlt\ \en{Personal Projects}\zh{个人项目}}
\datedsubsection{\textbf{\en{Gnar}\zh{Gnar}}}{\url{https://github.com/abcdlsj/gnar}}
\en{A tool similar to frp that supports subdomain forwarding}
\zh{内网端口转发工具,类似于 Frp/Ngork}
\begin{itemize}[parsep=0.5ex]
      \item \en{Forwarding protocol support TCP/UDP, Crontol connection using TCP, using Yamux support multiplexing}
            \zh{转发协议支持 TCP/UDP,Crontol 连接采用 TCP,使用 Yamux 支持多路复用}
\end{itemize}

\datedsubsection{\textbf{\en{cLSM}\zh{cLSM}}}{\url{https://github.com/abcdlsj/clsm}}
\en{An implementation of LSM-Tree data structure based on SkipList}
\zh{基于 SkipList 的 LSM-Tree 数据结构实现}
\begin{itemize}[parsep=0.5ex]
      \item \en{Memory level uses SkipList data structure, disk level uses ordered key-value pairs}
            \zh{内存层使用 SkipList 数据结构,磁盘层使用有序键值对}
\end{itemize}

\section{\faCogs\ \en{Skills}\zh{技能}}
\begin{itemize}[parsep=0.5ex]
      \item \en{\textbf{Programming Languages}:Familiar with Go foundation,C++,Python,Lua,familiar with common data structures and algorithms}
            \zh{\textbf{编程语言}:熟悉 Go,了解 C++、Python、Lua,熟悉常用数据结构与算法}
      \item \en{\textbf{System \& Network}:Processes,memory,Linux,multiplexing,containerization techniques,common network protocols,etc}
            \zh{\textbf{系统 \& 网络}:进程、内存、Linux、多路复用、容器化技术、常见网络协议等}
      \item \en{\textbf{Milleware Framework}:Knowledge of commonly used middleware (MySQL/Redis/Kafka/ElasticSearch/ClickHouse), ability to choose reasonable middleware according to business needs and implementation}
            \zh{\textbf{中间件}:了解常用中间件(MySQL/Redis/Kafka/ElasticSearch/ClickHouse),能根据业务需求选择合理的中间件并实现}
\end{itemize}

\section{\faInfo\ \en{Miscellaneous}\zh{杂项}}
\begin{itemize}[parsep=0.5ex]
      \item \en{Like to implement Small Side Project,includes Go Git,Tfidf wiki search engine,BitTorrent downloader,IM bot,Static-Blog generator,DNS server,line counter(Tokei/Scc),Proxy,Serverless Framework,Pastebin}
            \zh{喜欢实现 Small Side Project,包括 Go Git、Tfidf wiki 搜索引擎、BitTorrent downloader、IM bot、Static-Blog generator、line counter(Tokei/Scc)、Proxy、Serverless Framework、Pastebin 等等}
      \item \en{I love to explore, share and record.}
            \zh{喜欢折腾和探索,热爱分享和记录}
\end{itemize}

\end{document}