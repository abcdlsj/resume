\documentclass{resume}

\newcommand{\en}[1]{#1}
\newcommand{\zh}[1]{}

\zh{\usepackage{xeCJK}}
\zh{\setCJKmainfont{Noto Sans CJK SC}}
\zh{\setCJKsansfont{Noto Sans CJK SC}}
\zh{\setCJKmonofont{Noto Sans CJK SC}}

\hypersetup{hidelinks}

\begin{document}
\linespread{1.5}

\name{\en{Songjian Li}\zh{李松健}}
\basicInfo{
\email{lisongjianshuai@gmail.com} \textperiodcentered\
\homepage[Blog]{https://abcdlsj.github.io/} \textperiodcentered\
\github[GitHub]{https://github.com/abcdlsj/}
% \gitea[Gitea]{https://gitea.songjian.li}
}

\section{\faGraduationCap\ \en{Education}\zh{教育经历}}
\en{\datedsubsection{\textbf{HeFei University of Technology}}{09/2017 -- 2021/07}}
\zh{\datedsubsection{\textbf{合肥工业大学}}{2017/09 -- 2021/07}}
\begin{itemize}
      \item \en{Electronic Information Science and Technology, \textit{Computer College}}
            \zh{电子信息科学与技术(计算机学院)}
\end{itemize}

\en{\datedsubsection{\textbf{\href{https://www.sea.com/products/shopee/}{Shopee Inc.}}, ShenZheng, China}{07/2021 -- present}}
\zh{\datedsubsection{\textbf{\href{https://www.sea.com/products/shopee/}{深圳虾皮信息科技发展有限公司}}}{2021/07 -- 至今}}
\en{\role{Seller Platform}{Backend Software Engineer}}
\zh{\role{卖家平台}{后端软件工程师}}
\begin{itemize}[parsep=0.5ex]
      \item \en{}
            \zh{负责 Shopee 卖家主站首页(内容/弹窗/侧边栏等), Seller 通用 Feature Toggle(多地区黑白名单) 项目(项目 owner)}
      \item \en{}
            \zh{参与内部项目重构,参与组内其它需求的开发(内容包括但不限于 Oauth 对接、S3 数据迁移、网关插件编写 等等)}
      \item \en{}
            \zh{工作中能独立按需按时完成需求设计和实现}
\end{itemize}

\en{\role{Seller Open Platform}{Backend Software Engineer}}
\zh{\role{卖家开放平台(开发者平台)}{后端软件工程师}}
\begin{itemize}[parsep=0.5ex]
      \item \en{}
            \zh{负责 PartnerShip 项目(owner,类似 ToB 充值兑换券码平台)}
      \item \en{}
            \zh{基于 SellerGateway(Openresty)开发 Log-Report 插件,实现调用日志脱敏、上报等功能}
      \item \en{}
            \zh{参与开发平台页面后台(内容搜索、admin 端),参与内部中间件改造}
\end{itemize}

\section{\faGithubAlt\ \en{Work}\zh{工作项目}}
\datedsubsection{\textbf{Misc FrameWork}}{{}}
\en{Seller-Centre content module, pop-up module, sidebar, error code module, tag module and other generic modules for admin-side and user-side implementation}
\zh{Seller-Centre 首页}
\begin{itemize}
      \item \en{Developed using gin, it contains the home page Content/Popup/Sidebar/Settings, as well as the tag and errconfig modules}
            \zh{项目 owner,使用 gin 开发,包含首页内容、弹窗、侧边栏、错误码、标签等模块}
      \item \en{The design is divided into libraries and tables, and other modules are not coupled, with a clear division of functions}
            \zh{数据上分库分表,设计上功能划分清晰,业务拓展性强}
      \item \en{User side widely used asynchronous loading, caching, message queue active update implementation, improve response speed}
            \zh{广泛使用异步加载、缓存、消息 等方式提高用户体验}
\end{itemize}

\datedsubsection{\textbf{Seller Feature FrameWork}}{{}}
\en{Seller-Centre Feature Modules}
\zh{Seller-Centre Feature 模块}
\begin{itemize}
      \item \en{Used to restrict the display/operation of the seller's functions, providing HTTP and RPC interfaces}
            \zh{用于限制卖家的功能显示/操作,对外提供 HTTP 以及 RPC 查询接口}
      \item \en{The configuration side contains black and white list configuration, can use custom upload ID file configuration, can support the business side API configuration, but also through the Tag checkbox}
            \zh{配置端包含黑白名单配置,支持上传/勾选 Tag/API 等多种方式配置,有回滚/user 端通知/Tag 标记等功能}
      \item \en{Other teams are strongly dependent on the module, the load is large real-time requirements, the background through multi-processing to optimize performance}
            \zh{其它团队强依赖模块,负载大实时性要求强,后台通过多协程处理优化性能}
\end{itemize}

\datedsubsection{\textbf{Seller PartnerShip}}{{}}
\en{Seller PartnerShip}
\zh{Seller PartnerShip}
\begin{itemize}
      \item \en{}
            \zh{项目 owner,独立完成后端设计与开发,项目需对接多个团队}
      \item \en{}
            \zh{使用 gin 开发,包含充值/兑换/券码/订单/用户/定时任务/邮件通知等模块}
\end{itemize}

\section{\faGithubAlt\ \en{Portfolios}\zh{个人项目}}
\datedsubsection{\textbf{Pipe}}{\url{https://github.com/abcdlsj/pipe}}
\en{A domain support frp like tool}
\zh{支持子域的内网转发工具}
\begin{itemize}[parsep=0.5ex]
      \item \en{}
            \zh{使用 Go 实现,支持 TCP 协议,支持 Domain 配置}
\end{itemize}

\datedsubsection{\textbf{cLSM}}{\url{https://github.com/abcdlsj/clsm}}
\en{SkipList-based implementation of LSM-Tree data structure}
\zh{基于 SkipList 的 LSM-Tree 数据结构实现}
\begin{itemize}[parsep=0.5ex]
      \item \en{The Memory layer uses SkipList as the underlying layer and the Disk layer uses ordered key-value pairs as the underlying layer.}
            \zh{Memory 层采用 SkipList 作为底层,Disk 层使用有序键值对作为底层}
      \item \en{Use heap to optimize Merge operation, use BloomFilter to reduce the read amplification problem, and use tombstone mechanism to mark Key for deletion.}
            \zh{利用堆优化 Merge 操作,利用 BloomFilter 降低读放大问题,删除采用墓碑机制标记 Key}
      \item \en{cLSM circumvents the disk random write problem and dramatically improves write performance compared to B+ Tree at the expense of read performance}
            \zh{cLSM 规避磁盘随机写入问题,相比 B+ Tree 牺牲读性能,大幅提高写性能}
\end{itemize}

\section{\faCogs\ \en{Skills}\zh{技能}}
\begin{itemize}[parsep=0.5ex]
      \item \en{\textbf{Programming Languages}:Familiar with Go foundation, C++, Python, Lua and other languages, familiar with common data structures and algorithms}
            \zh{\textbf{编程语言}: 熟悉 Go 基础,了解 C++、Python、Lua 等语言,熟悉常用数据结构与算法}
      \item \en{\textbf{System}:Processes, memory, Linux I/O models, multiplexing, containerization techniques, common network protocols, etc.}
            \zh{\textbf{系统}: 进程、内存、Linux I/O 模型、多路复用、容器化技术、常见网络协议等}
      \item \en{\textbf{DataBase}:Familiar with MySQL database use,understand KV storage(LSM-Tree)}
            \zh{\textbf{数据库}: 熟悉 MySQL 数据库使用,了解 KV 存储(LSM-Tree)}
\end{itemize}

\section{\faInfo\ \en{Miscellaneous}\zh{杂项}}
\begin{itemize}[parsep=0.5ex]
      \item \en{}
            \zh{喜欢折腾,搭建过 Gitea、Minio、Grafana 等服务}
      \item \en{}
            \zh{热爱写代码和了解有趣的知识,喜欢实现小项目(手写Git,简单全文搜索引擎,简单 BitTorrent 下载器,Bot 工具,静态博客生成器,DNS 服务器,实现 Tokei 等等)}

\end{itemize}

\end{document}