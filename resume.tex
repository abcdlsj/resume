\documentclass{resume}

\newcommand{\en}[1]{#1}
\newcommand{\zh}[1]{}

\zh{\usepackage{xeCJK}}
\zh{\setCJKmainfont{Noto Sans CJK SC}}
\zh{\setCJKsansfont{Noto Sans CJK SC}}
\zh{\setCJKmonofont{Noto Sans CJK SC}}

\hypersetup{hidelinks}

\begin{document}
\linespread{1.5}

\name{\en{Songjian Li}\zh{李松健}}
\basicInfo{
\email{lisongjianshuai@gmail.com} \textperiodcentered\
\homepage[Blog]{https://blog.songjian.li/} \textperiodcentered\
\github[GitHub]{https://github.com/abcdlsj} \textperiodcentered\
\gitea[Gitea]{https://gitea.songjian.li}
}

\section{\faGraduationCap\ \en{Education}\zh{教育经历}}
\en{\datedsubsection{\textbf{HeFei University of Technology}, Undergraduate}{09/2017 -- 2021/07}}
\zh{\datedsubsection{\textbf{合肥工业大学}, 在读本科}{2017/09 -- 2021/07}}
\begin{itemize}
      \item \en{Electronic Information Science and Technology, \textit{Computer College}}
            \zh{电子信息科学与技术(计算机学院),2021 年毕业}
\end{itemize}

\section{\faUsers\ \en{Work Experience}\zh{工作经历}}
\en{\datedsubsection{\textbf{\href{https://www.didiglobal.com/}{DiDi Inc.}}, Beijing, China}{11/2020 -- 01/2021}}
\zh{\datedsubsection{\textbf{\href{https://www.didiglobal.com/}{北京嘀嘀无限科技发展有限公司}}}{2020/11 -- 2021/1}}
\en{\role{Business platform technology department}{R\&D Intern}}
\zh{\role{质量中台}{研发实习生}}
\begin{itemize}[parsep=0.5ex]
      \item \en{The main work is the maintenance and optimization of the internal quality testing platform (Go), as well as being responsible for the development of the internal testing platform and its version change (Python)}
            \zh{负责质量效能部门质量检测平台的维护(Go),以及内部测试流程框架的开发及其版本优化(Python)}
      \item \en{During the work period, I promoted the usage of the new version testing framework in many business lines, learned about CI/CD, and understood the use of development process platforms such as go-live and deployment}
            \zh{工作期间推进新版本测试框架在多条业务线的使用率,学习到关于 CI/CD 的知识,了解到上线、部署等开发流程平台的使用}
\end{itemize}

\en{\datedsubsection{\textbf{\href{https://www.sea.com/products/shopee/}{Shopee Inc.}}, ShenZheng, China}{07/2021 -- Present}}
\zh{\datedsubsection{\textbf{\href{https://www.sea.com/products/shopee/}{深圳虾皮信息科技发展有限公司}}}{2021/07 -- Present}}
\en{\role{Seller Platform}{Backend Software Engineer}}
\zh{\role{卖家平台}{后端软件工程师}}
\begin{itemize}[parsep=0.5ex]
      \item \en{Participate in the development of the Seller sub-account system, Gateway and other infrastructure}
            \zh{参与 Seller 主子账号系统、Gateway 等基础设施的开发}
      \item \en{Responsible for the development and optimization of the common framework of the Shopee Seller home page (content/pop-ups/sidebars), including the development and optimization of the admin side}
            \zh{负责 Shopee Seller 主站首页(内容/弹窗/侧边栏)等通用框架的开发和优化,包括 admin 端的开发优化}
      \item \en{Responsible for the development and optimization of Seller's generic Feature Toggle (multi-region black and white list), which is widely used and enhances the efficiency of the requirements online}
            \zh{负责 Seller 通用 Feature Toggle 的开发和优化(多地区黑白名单),使用广泛,提升了需求的上线效率}
      \item \en{Worked with Go (major/skilled), Python (old projects/efficiency scripts), Lua (Nginx/Openresty), etc., using Mysql, Redis/Codis, Kafka, Grafana, Prometheus, Jenkins, Docker, GitLab, etc.}
            \zh{工作期间使用 Go(主要/熟练)、Python(旧项目/效率脚本)、Lua(Nginx/Openresty)等,使用 Mysql、Redis/Codis、Kafka、Grafana、Prometheus、Jenkins、Docker、GitLab 等}
      \item \en{Work with caching, asynchronous, concurrency, flow limiting, fusing and other optimization methods, learn to use Go performance analysis tools, understand the use of Oauth and other authentication methods}
            \zh{工作中接触缓存、异步、并发、限流、熔断等优化方式,学会使用 Go 性能分析工具,了解使用 Oauth 等认证方式}
\end{itemize}

\section{\faGithubAlt\ \en{Work}\zh{工作项目}}
\datedsubsection{\textbf{Misc FrameWork}}{{}}
\en{Seller-Centre content module, pop-up module, sidebar, error code module, tag module and other generic modules for admin-side and user-side implementation}
\zh{Seller-Centre 内容模块、弹窗模块、侧边栏、错误码模块、标签模块以及其它通用模块的 admin 端和 user 端实现}
\begin{itemize}
      \item \en{Developed using gin, it contains the home page Content/Popup/Sidebar/Settings, as well as the tag and errconfig modules}
            \zh{使用 gin 开发,包含首页 Content/Popup/Sidebar/Settings,以及 tag、errconfig 模块}
      \item \en{The design is divided into libraries and tables, and other modules are not coupled, with a clear division of functions}
            \zh{设计上分库分表,与其它模块不耦合,功能划分清晰}
      \item \en{User side widely used asynchronous loading, caching, message queue active update implementation, improve response speed}
            \zh{user 端广泛使用异步加载、缓存、消息队列主动更新实现,提高响应速度}
\end{itemize}

\datedsubsection{\textbf{Seller Feature FrameWork}}{{}}
\en{Seller-Centre Feature Modules}
\zh{Seller-Centre Feature 模块}
\begin{itemize}
      \item \en{Used to restrict the display/operation of the seller's functions, providing HTTP and RPC interfaces}
            \zh{用于限制卖家的功能显示/操作,提供 HTTP 以及 RPC 接口}
      \item \en{The configuration side contains black and white list configuration, can use custom upload ID file configuration, can support the business side API configuration, but also through the Tag checkbox}
            \zh{配置端包含黑白名单配置,可使用自定义上传 ID 文件方式配置,可以支持通过业务方 API 方式配置,也可通过 Tag 勾选}
      \item \en{Other teams are strongly dependent on the module, the load is large real-time requirements, the background through multi-processing to optimize performance}
            \zh{其它团队强依赖模块,负载大实时性要求强,后台通过多协程处理优化性能}
\end{itemize}

\section{\faGithubAlt\ \en{Portfolios}\zh{个人项目}}
\datedsubsection{\textbf{cLSM}}{\url{https://github.com/abcdlsj/clsm}}
\en{SkipList-based implementation of LSM-Tree data structure}
\zh{基于 SkipList 的 LSM-Tree 数据结构实现}
\begin{itemize}[parsep=0.5ex]
      \item \en{The Memory layer uses SkipList as the underlying layer and the Disk layer uses ordered key-value pairs as the underlying layer.}
            \zh{Memory 层采用 SkipList 作为底层,Disk 层使用有序键值对作为底层}
      \item \en{Use heap to optimize Merge operation, use BloomFilter to reduce the read amplification problem, and use tombstone mechanism to mark Key for deletion.}
            \zh{利用堆优化 Merge 操作,利用 BloomFilter 降低读放大问题,删除采用墓碑机制标记 Key}
      \item \en{cLSM circumvents the disk random write problem and dramatically improves write performance compared to B+ Tree at the expense of read performance}
            \zh{cLSM 规避磁盘随机写入问题,相比 B+ Tree 牺牲读性能,大幅提高写性能}
\end{itemize}

\section{\faCogs\ \en{Skills}\zh{技能}}
\begin{itemize}[parsep=0.5ex]
      \item \en{\textbf{Programming Languages}:Familiar with Go foundation, C++, Python, Lua and other languages, familiar with common data structures and algorithms}
            \zh{\textbf{编程语言}: 熟悉 Go 基础,了解 C++、Python、Lua 等语言,熟悉常用数据结构与算法}
      \item \en{\textbf{System}:Processes, memory, Linux I/O models, multiplexing, containerization techniques, common network protocols, etc.}
            \zh{\textbf{系统}: 进程、内存、Linux I/O 模型、多路复用、容器化技术、常见网络协议等}
      \item \en{\textbf{DataBase}:Familiar with MySQL database use, understand KV storage (LSM-Tree)}
            \zh{\textbf{数据库}: 熟悉 MySQL 数据库使用,了解 KV 存储(LSM-Tree)}
\end{itemize}

\section{\faInfo\ \en{Miscellaneous}\zh{杂项}}
\begin{itemize}[parsep=0.5ex]
      \item \en{Open source enthusiast, like tossing love programming}
            \zh{开源爱好者,喜欢折腾热爱编程}
      \item \en{Build Self-Hosts services, Gitea, Minio, Grafana, Alerts Services, etc., implement Github Action/Telegram Bot/Git/Sreach-Engine and other small weekend projects}
            \zh{搭建 Self-Hosts 服务,Gitea、Minio、Grafana、Alerts Services 等,实现 Github Action/Telegram Bot/Git/Sreach-Engine 等周末小项目}
\end{itemize}

\end{document}