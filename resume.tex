\documentclass{resume}

\newcommand{\en}[1]{#1}
\newcommand{\zh}[1]{}

\zh{\usepackage{xeCJK}}
\zh{\setCJKmainfont{Noto Sans CJK SC}}
\zh{\setCJKsansfont{Noto Sans CJK SC}}
\zh{\setCJKmonofont{Noto Sans CJK SC}}

\hypersetup{hidelinks}
\linespread{1}
\setlength{\parskip}{0.3em}

\begin{document}

\name{\en{Songjian Li}\zh{李松健}}
\basicInfo{
      \email{career@songjian.li} \textperiodcentered\
      \phone{+86-17356375549} \textperiodcentered\
      \homepage[Blog]{https://abcdlsj.github.io/} \textperiodcentered\
      \github[GitHub]{https://github.com/abcdlsj/}
}

\section{\faGraduationCap\ \en{Education}\zh{教育经历}}
\datedsubsection{\textbf{\en{HeFei University of Technology}\zh{合肥工业大学}}}{09/2017 -- 07/2021}
\begin{itemize}
      \item \en{Bachelor's Degree in Electronic Information Science and Technology,\textit{School of Computer Science}}
            \zh{本科,电子信息科学与技术专业,\textit{计算机学院}}
\end{itemize}

\section{\faBriefcase\ \en{Work Experience}\zh{工作经历}}
\en{\datedsubsection{\textbf{\href{https://www.sea.com/products/shopee/}{Shopee Inc.}},ShenZheng,China}{07/2021 -- present}}
\zh{\datedsubsection{\textbf{\href{https://www.sea.com/products/shopee/}{深圳虾皮信息科技发展有限公司}}}{2021/07 -- 至今}}
\en{\role{Seller Platform \& Open Platform}{Backend Software Engineer}}
\zh{\role{卖家平台 \& Shopee 开放平台}{后端开发工程师}}
\begin{itemize}[parsep=0.6ex]
      \item \en{}
            \zh{开发和优化开放平台相关功能(鉴权、调用、延迟优化等);开发并维护卖家主站模块(弹窗、侧边栏、问卷系统、Admin RBAC 平台等)}
      \item \en{}
            \zh{参与多元化项目开发,包括重构、网关插件、Tracing接入、Common库、IOC框架接入等}
      \item \en{}
            \zh{使用 Go 语言,服务通信使用 RPC/HTTP,能熟练应用常用中间件和后台开发技术}
      \item \en{}
            \zh{具有独立设计和实施解决方案的能力,善于跨部门协作与沟通,有效管理项目进度并确保按时交付}
\end{itemize}

\section{\faCloud\ \en{Projects}\zh{工作项目}}

\datedsubsection{\textbf{\en{Open Service}\zh{Open Service}}}{\url{https://open.shopee.com/}}
\begin{itemize}[parsep=0.6ex]
      \item \en{Implemented OpenResty gateway plugin for developers to call the log report (field desensitization,log structure normalization),access to Shopee data platform to provide data source support for other functions}
            \zh{基于 OpenResty 网关实现调用上报插件(40k+ QPS),提升平台可观测性并为其它需求提供数据支持}
      \item \en{Added multiple functions of the user console (call Kanban,log search,business analytics,vulnerabilities board,etc) to enhance the functionality of the platform,improve the user experience and platform observability}
            \zh{为平台增加多功能,包括调用统计、商业分析、日志搜索、漏洞看板、推送优化等,提升用户体验和平台易用性}
\end{itemize}

\datedsubsection{\textbf{\en{Partner Voucher}\zh{Partner Voucher}}}{\url{https://partnervoucher.shopee.co.id/}}
\begin{itemize}[parsep=0.6ex]
      \item \en{Partner top-up voucher code management service, providing online top-up and exchange process, improving operation and user efficiency. Complete back-end design and development independently and coordinate with multiple teams to advance the project development part and manage the schedule}
            \zh{在线充值券码管理服务,提供在线充值兑换流程替代原有线下充值流程,提升充值体验和运营效率}
      \item \en{}
            \zh{独立完成后端设计与开发并与多个团队联调,负责推进项目开发部分并管理进度}
\end{itemize}

\datedsubsection{\textbf{\en{Seller Centre}\zh{Seller Centre}}}{\url{https://seller.shopee.co.id/}}
\begin{itemize}[parsep=0.6ex]
      \item \en{Contains the main home page content,pop-up windows and sidebars and other modules,widely used asynchronous loading,caching and message queues and other technologies,significantly improving the site's responsiveness and seller experience}
            \zh{主站服务包含首页内容、弹窗和侧边栏等模块,广泛采用了异步加载、缓存和消息队列等技术,显著提升了网站响应速度和卖家体验}
      \item \en{Feature toggle are used to flexibly control the display and limitation of functions,including multi-data source support,version rollback and multi-level query,etc.,which facilitates efficient management of operations}
            \zh{功能开关服务用于灵活控制功能的显示和限制,可配置项丰富,具备版本特性,便于运营高效管理}
\end{itemize}

\section{\faGithubAlt\ \en{Personal Projects}\zh{Personal}}
{\textbf{\en{Gnar}\zh{Gnar}}}\hfill{\url{https://github.com/abcdlsj/gnar}}
\begin{itemize}[parsep=0.6ex]
      \item \en{}
            \zh{内网端口转发工具,类似于 Frp/Ngork}
      \item \en{Forwarding protocol support TCP/UDPusing Yamux support multiplexing}
            \zh{转发协议支持 TCP/UDP,使用 Yamux 支持多路复用}
\end{itemize}

{\textbf{\en{cLSM}\zh{cLSM}}}\hfill{\url{https://github.com/abcdlsj/clsm}}
\begin{itemize}[parsep=0.6ex]
      \item \en{LSM-Tree is based on SkipList}
            \zh{基于 SkipList 的 LSM-Tree 数据结构实现,内存层使用 SkipList,磁盘层使用有序键值对}
\end{itemize}

\section{\faCogs\ \en{Skills}\zh{技能}}
\begin{itemize}[parsep=0.6ex]
      \item \en{\textbf{Programming Languages}:Familiar with Go foundation,C++,Python,Lua,familiar with common data structures and algorithms}
            \zh{\textbf{编程语言}:熟悉 Go,了解 C++、Python、Lua,熟悉常用数据结构与算法}
      \item \en{\textbf{System \& Network}:Processes,memory,Linux,multiplexing,containerization techniques,common network protocols,etc}
            \zh{\textbf{系统 \& 网络}:进程、内存、Linux、多路复用、容器化技术、常见网络协议等}
      \item \en{\textbf{Milleware Framework}:Knowledge of commonly used middleware (MySQL/Redis/Kafka/ElasticSearch/ClickHouse),ability to choose reasonable middleware according to business needs and implementation}
            \zh{\textbf{中间件}:了解常用中间件及组件(MySQL/Redis/Kafka/ElasticSearch/ClickHouse),能根据业务需求选择合理的中间件并接入}
\end{itemize}

\section{\faInfo\ \en{Miscellaneous}\zh{杂项}}
\begin{itemize}[parsep=0.6ex]
      \item \en{Like to implement Small Side Project,includes Go Git,Tfidf wiki search engine,BitTorrent downloader,IM bot,Static-Blog generator,DNS server,line counter(Tokei/Scc),Proxy(Frp/Ngork),Serverless Framework,Pastebin}
            \zh{喜欢实现 Small Side Project,包括 Go Git、Tfidf wiki 搜索引擎、BitTorrent downloader、IM bot、Static-Blog generator、line counter(Tokei/Scc)、Proxy(Frp/Ngork)、Serverless Framework、Pastebin 等等}
      \item \en{I love to explore,share and record.}
            \zh{喜欢折腾和探索,热爱分享和记录}
\end{itemize}

\end{document}