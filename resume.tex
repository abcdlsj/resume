\documentclass{resume}

\newcommand{\en}[1]{#1}
\newcommand{\zh}[1]{}

\zh{\usepackage{xeCJK}}
\zh{\setCJKmainfont{Noto Sans CJK SC}}
\zh{\setCJKsansfont{Noto Sans CJK SC}}
\zh{\setCJKmonofont{Noto Sans CJK SC}}

\hypersetup{hidelinks}

\begin{document}
\linespread{1.5}

\name{\en{Songjian Li}\zh{李松健}}
\basicInfo{
      \email{lisongjianshuai@gmail.com} \textperiodcentered\
      \phone{+86-13225639902} \textperiodcentered\
      \homepage[Blog]{https://abcdlsj.github.io/} \textperiodcentered\
      \github[GitHub]{https://github.com/abcdlsj/}
}

\section{\faGraduationCap\ \en{Education}\zh{教育经历}}
\datedsubsection{\textbf{\en{HeFei University of Technology}\zh{合肥工业大学}}}{09/2017 -- 07/2021}
\begin{itemize}
      \item \en{Bachelor of Electronic Information Science and Technology, \textit{Computer College}}
            \zh{本科,电子信息科学与技术(计算机学院)}
\end{itemize}

\section{\faBriefcase\ \en{Work Experience}\zh{工作经历}}
\en{\datedsubsection{\textbf{\href{https://www.sea.com/products/shopee/}{Shopee Inc.}}, ShenZheng, China}{07/2021 -- present}}
\zh{\datedsubsection{\textbf{\href{https://www.sea.com/products/shopee/}{深圳虾皮信息科技发展有限公司}}}{2021/07 -- 至今}}
\en{\role{Seller Platform/Openplatform}{Backend Software Engineer}}
\zh{\role{卖家平台/Shopee 开放平台(开发者平台)}{后端开发工程师}}
\begin{itemize}[parsep=0.5ex]
      \item \en{Developed and maintained Shopee seller main page (content/pop-up/sidebar, owner) and Seller Feature Toggle project (owner)}
            \zh{作为 Owner 开发和维护卖家主站首页(内容/弹窗/侧边栏)和卖家功能开关项目}
      \item \en{Developed Lua plugins for OpenResty gateway to implement call logging and JSON data anonymization}
            \zh{实现 OpenResty 网关的 OpenAPI 平台插件,实现 OpenAPI 调用日志上报和 JSON 脱敏规则,聚合归集数据后通过 ClickHouse 实现用户调用分析,通过 Elasticsearch 实现用户日志搜索}
      \item \en{Participated in internal project refactoring, development of other team requirements (including OAuth integration, S3 data migration, gateway plugin development, etc.), development of Openplatform pages, and internal middleware improvement}
            \zh{参与内部项目重构,开发团队需求(包括 OAuth 集成,S3 数据迁移,网关插件开发、改进内部中间件、Common 库维护等)}
      \item \en{Proficient in commonly used middleware, able to independently design and implement solutions according to requirements}
            \zh{熟练使用常用中间件,能够根据需求独立设计和实现解决方案}
\end{itemize}

\section{\faCloud\ \en{Projects}\zh{工作项目}}

\datedsubsection{\textbf{\en{Seller Centre}\zh{Seller Centre}}}{}
\begin{itemize}
      \item \en{Seller Center: Developed modules for the home page, pop-ups, sidebars, error codes, tags, and other common modules for admin and user interfaces using the Gin framework}
            \zh{使用 Gin 开发,包含主站首页内容、弹窗、侧边栏、错误码、标签等模块,以及 admin 端功能}
      \item \en{Implemented asynchronous loading, caching, and message queue update mechanisms on the user side to improve response speed}
            \zh{设计上功能划分清晰,业务拓展性强,广泛使用异步加载、缓存、消息 等方式提高用户体验}
\end{itemize}

\datedsubsection{\textbf{\en{Feature Toggle}\zh{Feature Toggle}}}{}
\begin{itemize}
      \item \en{Used for controlling the display/operation of seller features, providing HTTP and RPC interfaces}
            \zh{用于限制卖家的功能显示/操作,对外提供 HTTP/RPC 查询接口}
      \item \en{Includes configuration options such as blacklists/whitelists, custom upload ID file configuration, API configuration for business teams, and tag checkboxes}
            \zh{Souce 包括自定义上传、API 设置、预设 Tag 勾选等,支持黑白名单,包括回滚/user 端通知等功能}
      \item \en{Highly dependent on by other teams, required to handle a large load with real-time requirements. Optimized performance using multi-processing}
            \zh{卖家主站强依赖模块,负载大实时性要求强,提供不同级别外部接口}
\end{itemize}

\datedsubsection{\textbf{\en{Partner Voucher}\zh{Partner Voucher}}}{}
\begin{itemize}
      \item \en{Owner of the project, responsible for backend design and development, and coordination with multiple teams}
            \zh{独立完成后端设计与开发,与多个团队联调}
      \item \en{Developed using the Gin framework, it includes modules for recharge, redemption, coupon codes, orders, users, scheduled tasks, and email notifications}
            \zh{使用 Gin 开发,包含充值,兑换,券码,订单,用户,定时任务,邮件通知等模块}
\end{itemize}

\section{\faGithubAlt\ \en{Personal Projects}\zh{个人项目}}
\datedsubsection{\textbf{\en{Pipe}\zh{Pipe}}}{\url{https://github.com/abcdlsj/pipe}}
\en{A tool similar to frp that supports subdomain forwarding}
\zh{内网端口转发工具,类似于 Frp/Ngork}
\begin{itemize}[parsep=0.5ex]
      \item \en{Implemented in Go, supports TCP/UDP protocols}
            \zh{支持 TCP/UDP 协议}
      \item \en{Support multiplexing connections}
            \zh{支持 TCP 连接多路复用,使用 Yamux}
\end{itemize}

\datedsubsection{\textbf{\en{cLSM}\zh{cLSM}}}{\url{https://github.com/abcdlsj/clsm}}
\en{An implementation of LSM-Tree data structure based on SkipList}
\zh{基于 SkipList 的 LSM-Tree 数据结构实现}
\begin{itemize}[parsep=0.5ex]
      \item \en{Memory layer uses SkipList as the underlying data structure, and disk layer uses ordered key-value pairs}
            \zh{内存层使用 SkipList 数据结构,磁盘层使用有序键值对}
\end{itemize}

\section{\faCogs\ \en{Skills}\zh{技能}}
\begin{itemize}[parsep=0.5ex]
      \item \en{\textbf{Programming Languages}:Familiar with Go foundation, C++, Python, Lua and other languages, familiar with common data structures and algorithms}
            \zh{\textbf{编程语言}: 熟悉 Go 基础,了解 C++/Python/Lua,熟悉常用数据结构与算法}
      \item \en{\textbf{System}:Processes, memory, Linux, multiplexing, containerization techniques, common network protocols, etc.}
            \zh{\textbf{系统}: 进程、内存、Linux、多路复用、容器化技术、常见网络协议等}
      \item \en{\textbf{Milleware Framework}:Familiar with MySQL database use,understand KV storage(LSM-Tree)}
            \zh{\textbf{中间件}: 熟悉 MySQL/Redis,了解 ClickHouse/Elasticsearch,具备设计和优化数据库使用的能力}
\end{itemize}

\section{\faInfo\ \en{Miscellaneous}\zh{杂项}}
\begin{itemize}[parsep=0.5ex]
      \item \en{}
            \zh{喜欢折腾和探索,喜欢搭建 Self-hosted 服务}
      \item \en{}
            \zh{喜欢周末写点简单的项目,包括但不限于 Go Git, Tfidf wiki 搜索引擎,BitTorrent downloader,IM bot,Static-Blog generator,DNS server,line counter(Tokei/Scc) 等等}
\end{itemize}

\end{document}